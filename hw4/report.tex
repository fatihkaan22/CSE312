% Created 2021-06-26 Sat 20:16
% Intended LaTeX compiler: pdflatex
\documentclass[a4paper]{article}
\usepackage[utf8]{inputenc}
\usepackage[T1]{fontenc}
\usepackage{graphicx}
\usepackage{grffile}
\usepackage{longtable}
\usepackage{wrapfig}
\usepackage{rotating}
\usepackage[normalem]{ulem}
\usepackage{amsmath}
\usepackage{textcomp}
\usepackage{amssymb}
\usepackage{capt-of}
\usepackage{hyperref}
\documentclass{article}
\usepackage{here}
\usepackage{xcolor}
\usepackage[margin=3.0cm]{geometry}
\usepackage{amsmath}
\usepackage{parskip}
\renewcommand\arraystretch{1.4}
\usepackage[margin=1in]{geometry}
\usepackage{minted}
\usepackage{multicol}
\usepackage{tikz}
\usepackage{pgf}
\definecolor{bg}{rgb}{0.95,0.95,0.95}
\newminted{c}{fontsize=\footnotesize,frame=single,framesep=2mm}
\newminted{text}{fontsize=\footnotesize,frame=single,framesep=2mm,breaklines}
\author{Fatih Kaan Salgır - 171044009}
\date{}
\title{CSE312 - Operating Systems - HW\#4}
\hypersetup{
 pdfauthor={Fatih Kaan Salgır - 171044009},
 pdftitle={CSE312 - Operating Systems - HW\#4},
 pdfkeywords={},
 pdfsubject={},
 pdfcreator={Emacs 27.2 (Org mode 9.4.6)}, 
 pdflang={English}}
\begin{document}

\maketitle

\section*{Design Explanation}
\label{sec:orgc2cf438}

\quad

\begin{figure}[htbp]
\centering
\includegraphics[width=400px]{org-img/Design_Explanation/2021-06-25_17-11-06_screenshot.png}
\caption{Disk block alignments according to different block sizes.}
\end{figure}

\quad

Since there are lots of data blocks, data block area in the figure represents only 1\% of the whole data blocks in the disk.

\texttt{filesystem.h} has the functions, constans and variables that is used in both \texttt{makeFileSystem.c} and \texttt{fileSystemOper.c}. Global variables inside \texttt{filesystem.h};
\quad

\begin{ccode}
extern FILE *fp;
extern struct super_block super_blk;
extern uint32_t free_bitmap[BITMAP_WORD_SIZE];
extern uint12 fat_table[NO_BLOCKS];
\end{ccode}


\subsection*{Creating an empty file system}
\label{sec:org1d761bb}

To test \texttt{makeFileSystem}, disk file in examined with \texttt{hexdump} command in linux. First part is the  contents of the super block. The second is Free Block Bitmap is the first blocks of the bitmap is marked as full, because root directory is marked as full in bitmap. And linked one after another in FAT Table.

\begin{figure}[htbp]
\centering
\includegraphics[width=400px]{org-img/Creating_an_empty_file_system/2021-06-25_17-33-35_screenshot.png}
\caption{Testing \texttt{makeFileSystem} with 4KB block size}
\end{figure}

\newpage
\quad

\subsection*{Super Block}
\label{sec:orgc48c8e1}

Super block holds information about;
\begin{itemize}
\item block size,
\item number of blocks,
\item where the free bitmap, fat table root directory and data blocks starts and their sizes
\item number of free blocks
\end{itemize}

\begin{ccode}
// filesystem.h
struct super_block {
  uint32_t block_size;
  uint32_t block_count;
  uint32_t free_bitmap_start;
  uint32_t free_block_size;
  uint32_t no_free_blocks;
  uint32_t fat_table_start;
  uint32_t fat_table_block_size;
  uint32_t root_start;
  uint32_t root_size;
  uint32_t data_start;
  uint32_t data_size;
};
\end{ccode}



\subsection*{Free Blocks}
\label{sec:org69134af}

There are about 2\textsuperscript{12} data blocks (some of the blocks are reserved). In bitmap representation every data block is mapped to a bit. Therefore there must be 2\textsuperscript{12} bits in bitmap.
\begin{flalign*}
 & 2^{12} \mbox{ bits} = \frac{2^{12}}{2^{3}} \mbox{ bytes} = 512 \mbox{ bytes} &
\end{flalign*}
The bitmap fits inside a block, because the minimum block size is 512 bytes.
During implementation of the filesytem bit maniplulation is needed. 
Bitmap impelemented as \texttt{uint32\_t} array with size of 128 which is 2\textsuperscript{12} bits.

\texttt{get\_bit()}, \texttt{set\_bit()} and \texttt{clear\_bit()} functions are implemented to abstract away the implementation details.

\subsection*{FAT Table}
\label{sec:orgf8f3714}

There are \(2^{12}\) entries and every entry is 12 bits. Therefore the table size;
\begin{flalign*}
 & \frac{2^{12} \times 12}{2^{3}} = 6 \times 2^{10} \mbox{ Bytes} = 6 \mbox{ KB} &
\end{flalign*}
So the number of blocks needed for FAT table depends on the block size.
If the block size is 512 Bytes, then 12 blocks needed.
On the other hand if the block size is 4 KB, then 1.5 blocks will be enough. To align properly numbers completed to integers. Final table is;


\begin{table}[htbp]
\caption{Number of blocks needed for FAT table, according to different block size}
\centering
\begin{tabular}{l|l|r}
Block Size & FAT-12 & No blocks needed\\
\hline
\hline
0.5 KB & 2MB & 12\\
\hline
1 KB & 4MB & 6\\
\hline
2 KB & 8MB & 3\\
\hline
4 KB & 16MB & 2\\
\end{tabular}
\end{table}

\quad

When the file is long enough to exceed the capacity of a single block, a new available block will be found with help of bitmap. Current entry of the FAT Table linked to this new block. And the new block will get the value -1 (0xFFF).

Same thing holds when a directory run out of directory entries (Number of directory entry of a block can hold varies depending on the block size).

\begin{ccode}
// filesystem.h
struct uint12 {
  uint16_t x : 12;
};
typedef struct uint12 uint12;
extern uint12 fat_table[NO_BLOCKS];
\end{ccode}

\subsection*{Directory Entry}
\label{sec:orgfdd6a1f}

32 bytes used for each entry. Address is the block number of the first block. Attributes shows if the entry is file or directory. If the entry is a file then size shows how many bytes the file size. However if the entry is a directory, the size shows how many files/folders inside the directory. 

Size of the directory can be reperesented with 3 bytes, since it doesn't exceed maximum file size in the file system. In order to store the size in 3 bytes, array with size 3 and \texttt{uint8\_t} type is used. \texttt{pack3b()} and \texttt{unpack3b()} functions are used to fit 32 bit integer into 24 bits.

\quad

\begin{figure}[htbp]
\centering
\includegraphics[width=400px]{org-img/Design_Explanation/2021-06-25_17-12-23_screenshot.png}
\caption{Directory entries - number of bytes used}
\end{figure}

\newpage

\begin{ccode}
struct dir_entry {
  char name[20];           // 20 bytes
  uint16_t address;        // 2 bytes FAT-12 (used only least significant 12 bits)
  uint8_t attr;            // 1 byte
  uint8_t size[3];         // 3 bytes : max 16 MB size
  uint8_t time[3];         // 3 bytes : HMS
  uint8_t date[3];         // 3 bytes : DMY (Y: 1900+)
} __attribute__((packed)); // required for struct to be exactly 32 bytes

typedef struct dir_entry dir_entry;
\end{ccode}


\section*{Running \& Test Results}
\label{sec:org572fa0f}

File system manipulation functions are implemented in \texttt{fileSystemOper.c}.

\begin{table}[htbp]
\caption{Function names for corresponding operations}
\centering
\begin{tabular}{l|l|l}
Operation & Function & Notes\\
\hline
\hline
dir & \texttt{fs\_dir()} & - prints error if the target is not a directory\\
\hline
mkdir & \texttt{fs\_mkdir()} & - prints error if the prior path doesn't exist\\
 &  & - prints error if the directory already exist\\
\hline
rmdir & \texttt{fs\_rmdir()} & - prints error if the target is not a directory\\
 &  & - prints error if the directory has files/directories inside\\
\hline
dumpe2fs & \texttt{dumpe2fs()} & - prints superblock, free bitmap, occupied blocks\\
\hline
write & \texttt{fs\_write()} & - if file exists in the file system, overwrites\\
 &  & - if the linux file doesn't exist, prints error\\
\hline
read & \texttt{fs\_read()} & - if the linuxfile exits, overwrites\\
\hline
del & \texttt{fs\_del()} & - if the target is a directory, it will print error\\
\end{tabular}
\end{table}

\newpage

Test cases in the assignment pdf:

\begin{itemize}
\item Long file content is used to demonstrate multiple number of blocks used for a single file.
\item Middle part of the \texttt{dumpe2fs} is trunctaed for readibility.
\end{itemize}


\begin{textcode} 
> ./makeFileSystem 0.5 mySystem.data
> ./fileSystemOper mySystem.data mkdir '\usr'
> ./fileSystemOper mySystem.data mkdir '\usr\ysa'
> ./fileSystemOper mySystem.data mkdir '\bin\ysa'
ERROR: Path is incorrect, couldn't find: bin
> echo 'Lorem ipsum dolor sit amet, consetetur sadipscing elitr, sed diam nonumy eirmod tempor invidunt ut labore et dolore magna aliquyam erat, sed diam voluptua. At vero eos et accusam et justo duo dolores et ea rebum. Stet clita kasd gubergren, no sea takimata sanctus est Lorem ipsum dolor sit amet.

Lorem ipsum dolor sit amet, consetetur sadipscing elitr, sed diam nonumy eirmod tempor invidunt ut labore et dolore magna aliquyam erat, sed diam voluptua. At vero eos et accusam et justo duo dolores et ea rebum. Stet clita kasd gubergren, no sea takimata sanctus est Lorem ipsum dolor sit amet.' > linuxFile.data
> ./fileSystemOper mySystem.data write '\usr\file2' linuxFile.data
> ./fileSystemOper mySystem.data write '\file3' linuxFile.data
> ./fileSystemOper mySystem.data dir '\'
26-05-2021  19:18  <DIR>         usr
26-05-2021  19:18           594  file3
         1 File(s)        594 bytes
         1 Dir(s)    2076672 bytes free
> ./fileSystemOper mySystem.data del '\usr\ysa\file1'
> ./fileSystemOper mySystem.data dump2fs
==== SUPERBLOCK ====
Block Size:            512
Block Count:           4096
Free Bitmap Start:     512
Free Blocks Size:      4096
Numer of Free Blocks:  4057
FAT Table Start:       4608
FAT Table Block Size:  12
Root Start:            12800
Root Size:             7168
Data Start:            19968
Data Size:             2080768
====== BITMAP ======
    0 : 1111111111111111 1101111100000000 0000000000000000 0000000000000000
   64 : 0000000000000000 0000000000000000 0000000000000000 0000000000000000
  128 : 0000000000000000 0000000000000000 0000000000000000 0000000000000000
  192 : 0000000000000000 0000000000000000 0000000000000000 0000000000000000
...(truncated)
 3968 : 0000000000000000 0000000000000000 0000000000000000 0000000000000000
 4032 : 0000000000000000 0000000000000000 0000000000000000 0000000000000000
====== FILENAME : OCCUPIED BLOCKS ======
usr             : 16 
ysa             : 17 
file2           : 20 21 
file3           : 22 23 
Total number of directories: 2
Total number of files: 2
> ./fileSystemOper mySystem.data read '\usr\file2' linuxFile2.data
> cmp linuxFile2.data linuxFile.data
\end{textcode} 
\end{document}